\documentclass{article}

\usepackage{fancyhdr}
\usepackage{extramarks}
\usepackage{amsmath}
\usepackage{amsthm}
\usepackage{amsfonts}
\usepackage{tikz}
\usepackage{csquotes}
\usepackage[plain]{algorithm}
\usepackage{algpseudocode}
\usepackage[spanish]{babel}
\usepackage[
backend=bibtex,
style=alphabetic,
sorting=ynt
]{biblatex}
\addbibresource{bibl.bib}
\usepackage{float}
\restylefloat{table}
\usepackage{hyperref}
\usepackage[utf8]{inputenc}

\usetikzlibrary{automata,positioning}

%
% Basic Document Settings
%

\topmargin=-0.45in
\evensidemargin=0in
\oddsidemargin=0in
\textwidth=6.5in
\textheight=9.0in
\headsep=0.25in

\linespread{1.1}

\pagestyle{fancy}
\lhead{\hmwkClass}
\chead{\hmwkTitle}
\rhead{\firstxmark}
\lfoot{\lastxmark}
\cfoot{\thepage}

\renewcommand\headrulewidth{0.4pt}
\renewcommand\footrulewidth{0.4pt}

\setlength\parindent{0pt}

%
% Create Problem Sections
%

\newcommand{\enterProblemHeader}[1]{
    \nobreak\extramarks{}{Problema \arabic{#1} continua en la p\'agina siguiente\ldots}\nobreak{}
    \nobreak\extramarks{Problema \arabic{#1} (continuaci\'on)}{Problema \arabic{#1} continua en la p\'agina siguiente\ldots}\nobreak{}
}

\newcommand{\exitProblemHeader}[1]{
    \nobreak\extramarks{Problema \arabic{#1} (continuaci\'on)}{Problema \arabic{#1} continua en la p\'agina siguiente\ldots}\nobreak{}
    \stepcounter{#1}
    \nobreak\extramarks{Problema \arabic{#1}}{}\nobreak{}
}

\setcounter{secnumdepth}{0}
\newcounter{partCounter}
\newcounter{homeworkProblemCounter}
\setcounter{homeworkProblemCounter}{1}
\nobreak\extramarks{Problema \arabic{homeworkProblemCounter}}{}\nobreak{}

%
% Homework Problem Environment
%
% This environment takes an optional argument. When given, it will adjust the
% problem counter. This is useful for when the problems given for your
% assignment aren't sequential. See the last 3 problems of this template for an
% example.
%
\newenvironment{homeworkProblem}[1][-1]{
    \ifnum#1>0
        \setcounter{homeworkProblemCounter}{#1}
    \fi
    \section{Problema \arabic{homeworkProblemCounter}}
    \setcounter{partCounter}{1}
    \enterProblemHeader{homeworkProblemCounter}
}{
    \exitProblemHeader{homeworkProblemCounter}
}

%
% Homework Details
%   - Title
%   - Due date
%   - Class
%   - Section/Time
%   - Instructor
%   - Author
%

\newcommand{\hmwkTitle}{Tarea\ \#2}
\newcommand{\hmwkDueDate}{Noviembre 11, 2020}
\newcommand{\hmwkClass}{Estructuras Discretas}
\newcommand{\hmwkClassTime}{}
\newcommand{\hmwkClassInstructor}{Profesora Alma Ar\'evalo Loyola}
\newcommand{\hmwkAuthorName}{\textbf{Sof\'ia Guadalupe Alatorre Méndez} \and \textbf{Diego Navarro Macías} \and \textbf{Edson Jair Morales Hernández}}


%
% Title Page
%

\title{
    \vspace{2in}
    \textmd{\textbf{\hmwkClass:\ \hmwkTitle}}\\
    \normalsize\vspace{0.1in}\small{Entregar\ el\ \hmwkDueDate\ a las 11:59pm}\\
    \vspace{0.1in}\large{\textit{\hmwkClassInstructor\ \hmwkClassTime}}
    \vspace{3in}
}


\author{\hmwkAuthorName}
\date{}

\renewcommand{\part}[1]{\textbf{\large Parte \Alph{partCounter}}\stepcounter{partCounter}\\}

%
% Various Helper Commands
%

% Useful for algorithms
\newcommand{\alg}[1]{\textsc{\bfseries \footnotesize #1}}

% For derivatives
\newcommand{\deriv}[1]{\frac{\mathrm{d}}{\mathrm{d}x} (#1)}

% For partial derivatives
\newcommand{\pderiv}[2]{\frac{\partial}{\partial #1} (#2)}

% Integral dx
\newcommand{\dx}{\mathrm{d}x}

% Alias for the Solution section header
\newcommand{\solution}{\textbf{\large Solución}}

% Probability commands: Expectation, Variance, Covariance, Bias
\newcommand{\E}{\mathrm{E}}
\newcommand{\Var}{\mathrm{Var}}
\newcommand{\Cov}{\mathrm{Cov}}
\newcommand{\Bias}{\mathrm{Bias}}

\begin{document}


\maketitle

\pagebreak

\begin{homeworkProblem}
¿Qué es la lógica?\\
\\
\solution\\
\\
La lógica es la disciplina la cual se encarga del estudio del razonamiento, por medio de reglas y técnicas determina si un argumento es válido o no.
\end{homeworkProblem}

\begin{homeworkProblem}
Menciona un ejemplo del uso de la lógica en Ciencias de la Computación.\\
\\
\solution\\
\\
El nivel menos abstracto dentro de una computadora está constituido por circuitos electrónicos que responden a diferentes señales eléctricas, siguiendo los patrones de la \textbf{lógica booleana}, estas compuertas lógicas devuelven un valor dependiendo de las entradas que le dan al sistema. Existen 8 compuertas lógicas, las cueles son: AND, OR, Inverter, Buffer, NAND, NOR, XOR y XNOR.
\end{homeworkProblem}

\begin{homeworkProblem}
¿Qué es una proposición?\\
\\
\solution\\
\\
Es un enunciado del cual se puede conocer su valor de verdad, es decir; podemos determinar si es verdadero o falso.
Con el contexto se puede determinar si un enunciado es una proposición, si no hay contexto entonces no es una proposición y las oraciones imperativas no son proposiciones.
\end{homeworkProblem}

\begin{homeworkProblem}
¿Qué es un predicado?\\
\\
\solution\\
\\
Un predicado es una expresión lingüística que puede conectarse con una o varias otras expresiones para formar una oración. Por ejemplo, en la oración "Marte es un planeta", la expresión "es un planeta" es un predicado que se conecta con la expresión "Marte" para formar una oración; entonces, un predicado expresara la(s) propiedad(es) que cumple el sujeto.
Los predicados en lógica matemática también son tratados como funciones
\end{homeworkProblem}

\pagebreak

\begin{homeworkProblem}
¿Qué es un argumento lógico?\\
\\
\solution\\
\\
Un argumento lógico es una secuencia de proposiciones seguidos de una conclusión. Y un argumento es considerado correcto o válido si es una tautología.
\end{homeworkProblem}

\begin{homeworkProblem}
¿Qué es una tautología?\\
\\
\solution\\
\\
Es una fórmula que siempre se evalúa como verdadera, para todas las posibles asignaciones de sus variables (la columna resultado de su tabla de verdad tiene sólo unos).
\end{homeworkProblem}

\begin{homeworkProblem}
¿Qué es una contradicción?\\
\\
\solution\\
\\
Es cuando una fórmula se evalúa como falsa en todos sus estados posibles (la columna resultado de su tabla de verdad tiene sólo ceros).
\end{homeworkProblem}

\begin{homeworkProblem}
¿Qué es una contingencia? (En Lógica)\\
\\
\solution\\
\\
Es cuando una fórmula no es tautología ni contradicción.
\end{homeworkProblem}

\begin{homeworkProblem}
¿Qué es la equivalencia? (En lógica)\\
\\
\solution\\
\\
Dos proposiciones o fórmulas lógicas son lógicamente equivalentes cuando toman el mismo valor de verdad en todas las posibles asignaciones de valores de sus variables, es decir; las tablas de verdad de las dos proposiciones dan el mismo resultado.
\end{homeworkProblem}

\pagebreak

\begin{homeworkProblem}
¿Qué es un modelo? (En lógica)\\
\\
\solution\\
\\
Un modelo de un lenguaje formal es una estructura donde las oraciones formales de la teoría (es decir, una cadena de signos matemáticos) son interpretables y por tanto las oraciones pueden considerarse como afirmaciones sobre el modelo. Si un modelo para un lenguaje formal satisface además una oración o una teoría (conjunto de oraciones), se llama modelo de una oración o teoría.\\
La idea es que un lenguaje formal es simplemente un inventario de los signos que vamos a usar, y un modelo de un lenguaje formal es una asignación de significado a cada uno de esos signos.
\end{homeworkProblem}

\begin{homeworkProblem}
¿Qué es una premisa?\\
\\
\solution\\
\\
Una premisa es cada una de las proposiciones anteriores a la conclusión de argumento.En un argumento válido, las premisas implican la conclusión, pero esto no es necesario para que una proposición sea una premisa: lo único relevante es su lugar en el argumento, no su rol. Al ser proposiciones, las premisas siempre afirman o niegan algo que puede ser verdadero o falso.
\end{homeworkProblem}

\begin{homeworkProblem}
¿Qué es una conclusión?\\
\\
\solution\\
\\
Una conclusión es una proposición al final de un argumento, después de las premisas. Con los criterios adecuados e pude afirmar que si las premisas son verdaderas entonces la conclusión también tiene que serlo necesariamente.
\end{homeworkProblem}

\pagebreak

\begin{homeworkProblem}
Da ejemplo de 5 expresiones (en lenguaje natural) que sean proposiciones y 3 que no lo sean.\\
\\
\solution\\
\\
Expresiones que son proposiciones:
\begin{enumerate}
    \item Cuatro es un número par.
    \item El rombo es un equilátero.
    \item Los leones no son voluminosos.
    \item Hoy es miércoles.
    \item La Tierra es plana.
\end{enumerate}
Expresiones que \textbf{no} son proposiciones:
\begin{enumerate}
    \item ¡Hola!
    \item Limpia la escalera.
    \item ¿Qué cayo del cielo?
\end{enumerate}
\end{homeworkProblem}

\begin{homeworkProblem}
Identifica en los siguientes enunciados cuáles son las proposiciones
atómicas y tradúcelas asignando utilizando variables proposicionales y los operadores que sean necesarios:

  \begin{enumerate}
      \item El libro es rojo.
      \item Juan y Mariana tienen sed.
      \item Como está muy nublado, va a llover. Por lo tanto no saldremos.
      \item El lápiz es rojo o es amarillo.
  \end{enumerate}

\end{homeworkProblem}

\begin{homeworkProblem}
 Menciona los operadores lógicos que cumplen con la propiedad conmutativa justificando tu respuesta. \\
\\
\solution\\
\\
Los operadores logicos que cumplen con esta propiedad son la conjunción: $\wedge$, la disyunción: $\vee$ y la doble implicación: $\longleftrightarrow$; por ejemplo supongamos que expresar $p \wedge q$, es lo mismo que expresar $q \wedge p$, lo mismo para la disyunción y la doble implicación, tenemos $p \vee q$ es igual a $q \vee p$, y por ultimo $p \longleftrightarrow q$ es igual a $q \longleftrightarrow p$
\end{homeworkProblem}

\begin{homeworkProblem}
Menciona los operadores lógicos que cumplen con la propiedad asociativa justificando tu respuesta. \\
\\
\solution\\
\\
Los operadores logicos que vendrían cumpliendo esta propiedad son los mismos que del problema anterior y sabemos que la asociatividad es de la forma: $(a + b)+c$ = $a + (b + c)$.
\\
En justificación de nuestra respuesta, si tenemos a $(p \wedge q) \wedge r$ es lo mismo que expresar $p \wedge (q \wedge r)$, y si tenemos $(p \vee q) \vee r$ es igual que $p \vee (q \vee r)$, y aplicamos lo mismo para la doble implicación $(p \Longleftrightarrow q) \Longleftrightarrow r$ $=$ $p \Longleftrightarrow (q \Longleftrightarrow r)$


\end{homeworkProblem}

\begin{homeworkProblem}[34]
	Demuestra que con los operadores $\vee$ y $\neg$ es posible construir las tablas
	de verdad de los operadores $\wedge, \longrightarrow, \longleftrightarrow$
\end{homeworkProblem}
\begin{homeworkProblem}
  Traduce los siguientes argumentos a lógica proposicional y analiza
  si existe un modelo para cada uno de los argumentos:
  \begin{enumerate}
      \item Si se admite la teoría del eterno retorno, se debe admitir la existencia de
      entidades corpusculares identficables a través del tiempo y que se pueda hablar de un
      estado del universo definido en cada instante individual. Ahora bien, no es cierto que
      haya entidades corpusculares permanentes y estados del universo definifo en cada
      instante. Por tanto, la teoría del eterno retorno es inadmisible.\\
      $p$: Si se admite la teoría del eterno retorno...\\
      $q$: Si se admite la existencia de entidades...\\
      $r$: Se habla de un estado del universo...
      \item Si los habitantes de Venus invaden la Tierra, entonces los hombres se pondrán
      nerviosos o las mujeres se entusiasmarán. si los hombres se ponen nerviosos, las
      mujeres se entusiasmarán. Por tanto, si los habitantes de Venus invaden la Tierra,
      las mujeres se entusiasmarán.\\
      $p$: Los habitantes de Venus invaden la Tierra.\\
      $q$: Los hombres se pondrán nerviosos.\\
      $r$: Las mujeres se entusiasmarán.\\
      \item Si las autoridades prohiben fumar en pipa a los feos, entonces los grupos se
      alzarán indignados porque no venden pipas. Si los guapos no venden pipas o las
      autoridades no crean nuevos puestos de trabajo, entonces la nación no saldrá de la crisis
      económica. La nación no sale de la crisis económica y los guapos no venden pipas. Por
      lo tanto, las autoridades no prohibirán fumar en pipa a los feos.\\
      $p$: Las autoridades prohiben fumar en pipa a los feos\\
      $q$: Los guapos se alzarán\\
      $r$: Los guapos venden pipas\\
      $s$: Las autoridades crean nuevos empleos\\
      $t$: La nación saldrá de crisis\\
  \end{enumerate}
\end{homeworkProblem}
\begin{homeworkProblem}
  Niega de manera correcta las siguientes expresiones:
      \begin{itemize}
      \item $\forall x\ (x \leq y \longrightarrow \exists z\ (x + z = y))$
      \item $\forall \epsilon > 0\  \exists \delta > 0 \ (|x-y|<\delta \longrightarrow
      |f(x) -f(y)| < \epsilon)$
      \item $|x - y + z| = 4\longrightarrow  \exists y\ \forall z\ R(y,z)$
      \item $\forall x \in \mathbf{N}\ \exists y \in \mathbf{N}\ suc(x) = y$
  \end{itemize}
\end{homeworkProblem}
\begin{homeworkProblem}
  En las siguientes proposiciones identifica las variables libres, las
    variables ligadas y los alcances de los cuantificadores. No omitas
    el introducir paréntesis donde sea necesario.
    \begin{itemize}
        \item $\forall x\ x \leq y \longrightarrow \exists z\ x + z = y$
        \item $\forall \epsilon > 0\  \exists \delta > 0 (|x-y|<\delta \longrightarrow
        |f(x) -f(y)| < \epsilon$)
        \item $\forall x \in \mathbf{N}\ \exists y \in \mathbf{N}\ suc(x) = y$
        \item $\exists x \forall y \exists a(z + 2w = 4y \vee 2s = y^3+7)$
        \item $\forall \epsilon(\epsilon > 0 \longrightarrow \exists \delta > 0 \vee \epsilon + y = x)$
        \item $\forall x \forall y \exists z R(x,n) \longrightarrow P(z)$
        \item $P(0) \wedge \forall k \ (P(k) \longrightarrow P(suc(k))) \longrightarrow \forall n  P(n)$
    \end{itemize}
\end{homeworkProblem}
\begin{homeworkProblem}
    Determina si las siguientes fórmulas son satisfacibles. Da un modelo para
    cada fórmula que sí lo sea.
        \begin{enumerate}
            \item $p \wedge q \longleftrightarrow \neg p \wedge q$
            \item $(\neg p \vee q) \wedge p$
            \item $p \wedge q \wedge \neg p$
            \item $(p \longrightarrow q) \wedge (q \longrightarrow p)$
        \end{enumerate}
\end{homeworkProblem}
\begin{homeworkProblem}
    Determina si los siguientes conjuntos son satisfacibles.
        \begin{enumerate}
            \item $\Gamma = \{(\neg q \wedge r) \vee p \vee q, p \wedge r\}$
            \item $\Gamma = \{p \wedge \neg q, \neg(q \vee \neg p)\}$
            \item $\Gamma = \{q \vee r \vee s, \neg(q \vee r), \neg(r \vee s), \neg(s \vee q)\}$
            \item $\Gamma = \{\neg(p \wedge q) \wedge \neg(p \wedge r), q \vee r, \neg(p \vee \rightarrow \neg r)\}$
        \end{enumerate}
\end{homeworkProblem}
\begin{homeworkProblem}
  Entras a trabajar a una empresa que se dedica a fabricar
  proposiciones lógicas y esa empresa se jacta de que todas sus proposiciones
  son tautologías. Un día estás en una junta con tu jefe y descubren
  que el computólogo anterior que se encargaba de diseñar las proposiciones
  no se aseguró de que todas las proposiciones que hizo fuesen tautologías
  \footnote{No llevó el curso de Estructuras discretas con Alma.}, lo
  cual va en contra del eslogan de la empresa y temen que la auditoría de la
  LOGI-Profeco les levante una sanción. Tu jefe te encarga ``arreglar" las
  proposiciones, con dos condiciones:
  \begin{itemize}
      \item No puedes introducir nuevas variables lógicas en las proposiciones
      \item Si vas a introducir operadores nuevos, la proposición
      resultante puede exceder en a lo más el doble de operadores de la proposición
      original
      \item (Opcional) Si garantizas que toda proposición lógica la arreglas en menos de
      5 pasos te vuelves el Jefe de la División de Lógica de la empresa
  \end{itemize}
  El test que realiza el auditor de LOGI-Profeco consiste en tomar una
  asignación de valores de verdad aleatoria para el conjunto de variables
  de la proposición en cuestión (de ahí la razón de que no puedas agregar
  nuevas variables, ya que el test lo detectaría y automáticamente fallaría)
  y evaluaría la formula. Describe el proceso que realizarías para
  solucionar este problema.
\end{homeworkProblem}

\begin{homeworkProblem}

    ¿Cuántos renglones tiene la tabla de verdad de una proposición de $n$
    variables? Justifica tu respuesta. (Si das una demostración formal se te da
    un punto extra).
\end{homeworkProblem}

\begin{homeworkProblem}    
    Pasa las siguientes expresiones a forma normal disyuntiva.
        \begin{enumerate}
            \item $(p \longrightarrow q) \vee (q \longrightarrow p)$
            \item $(p \longrightarrow r) \wedge (q \longrightarrow r)$
            \item $(\neg r) \longrightarrow (p \longrightarrow q)$
            \item $(\neg p) \longleftrightarrow q$
        \end{enumerate}
\end{homeworkProblem}

\pagebreak

\cite{miquelteorias,lodepe}
\printbibliography

%\section{Bibliografía}
%\begin{itemize}
%\item https://www.ecured.cu/Logicamatematica % me manda 404
%\item https://es.wikipedia.org/wiki/PuertalConjuntodepuertasl % no encontre la pagina de esto
%\item https://sites.google.com/site/mathematicasdiscretesolutions/logica-de-po
%\item https://www.fing.edu.uy/~amiquel/fundamentos/teoriasymodelos.pdf
%\item https://es.wikipedia.org/wiki/Premisa\#:\~ :text=Enlogicaunapremisaelargumento
%\end{itemize}

\end{document}