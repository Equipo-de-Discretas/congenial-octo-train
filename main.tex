\documentclass{article}

\usepackage{fancyhdr}
\usepackage{extramarks}
\usepackage{amsmath}
\usepackage{amsthm}
\usepackage{amsfonts}
\usepackage{tikz}
\usepackage[plain]{algorithm}
\usepackage{algpseudocode}
\usepackage[spanish]{babel}
\usepackage[
backend=bibtex,
style=alphabetic,
sorting=ynt
]{biblatex}
\addbibresource{bibl.bib}

\usetikzlibrary{automata,positioning}

%
% Basic Document Settings
%

\topmargin=-0.45in
\evensidemargin=0in
\oddsidemargin=0in
\textwidth=6.5in
\textheight=9.0in
\headsep=0.25in

\linespread{1.1}

\pagestyle{fancy}
\lhead{\hmwkClass}
\chead{\hmwkTitle}
\rhead{\firstxmark}
\lfoot{\lastxmark}
\cfoot{\thepage}

\renewcommand\headrulewidth{0.4pt}
\renewcommand\footrulewidth{0.4pt}

\setlength\parindent{0pt}

%
% Create Problem Sections
%

\newcommand{\enterProblemHeader}[1]{
    \nobreak\extramarks{}{Problema \arabic{#1} continua en la p\'agina siguiente\ldots}\nobreak{}
    \nobreak\extramarks{Problema \arabic{#1} (continuaci\'on)}{Problema \arabic{#1} continua en la p\'agina siguiente\ldots}\nobreak{}
}

\newcommand{\exitProblemHeader}[1]{
    \nobreak\extramarks{Problema \arabic{#1} (continuaci\'on)}{Problema \arabic{#1} continua en la p\'agina siguiente\ldots}\nobreak{}
    \stepcounter{#1}
    \nobreak\extramarks{Problema \arabic{#1}}{}\nobreak{}
}

\setcounter{secnumdepth}{0}
\newcounter{partCounter}
\newcounter{homeworkProblemCounter}
\setcounter{homeworkProblemCounter}{1}
\nobreak\extramarks{Problema \arabic{homeworkProblemCounter}}{}\nobreak{}

%
% Homework Problem Environment
%
% This environment takes an optional argument. When given, it will adjust the
% problem counter. This is useful for when the problems given for your
% assignment aren't sequential. See the last 3 problems of this template for an
% example.
%
\newenvironment{homeworkProblem}[1][-1]{
    \ifnum#1>0
        \setcounter{homeworkProblemCounter}{#1}
    \fi
    \section{Problema \arabic{homeworkProblemCounter}}
    \setcounter{partCounter}{1}
    \enterProblemHeader{homeworkProblemCounter}
}{
    \exitProblemHeader{homeworkProblemCounter}
}

%
% Homework Details
%   - Title
%   - Due date
%   - Class
%   - Section/Time
%   - Instructor
%   - Author
%

\newcommand{\hmwkTitle}{Tarea\ \#2}
\newcommand{\hmwkDueDate}{Noviembre 11, 2020}
\newcommand{\hmwkClass}{Estructuras Discretas}
\newcommand{\hmwkClassTime}{}
\newcommand{\hmwkClassInstructor}{Profesora Alma Ar\'evalo Loyola}
\newcommand{\hmwkAuthorName}{\textbf{Sof\'ia Guadalupe Alatorre Méndez} \and \textbf{Diego Navarro Macías} \and \textbf{Edson Jair Morales Hernández}}


%
% Title Page
%

\title{
    \vspace{2in}
    \textmd{\textbf{\hmwkClass:\ \hmwkTitle}}\\
    \normalsize\vspace{0.1in}\small{Entregar\ el\ \hmwkDueDate\ a las 11:59pm}\\
    \vspace{0.1in}\large{\textit{\hmwkClassInstructor\ \hmwkClassTime}}
    \vspace{3in}
}


\author{\hmwkAuthorName}
\date{}

\renewcommand{\part}[1]{\textbf{\large Parte \Alph{partCounter}}\stepcounter{partCounter}\\}

%
% Various Helper Commands
%

% Useful for algorithms
\newcommand{\alg}[1]{\textsc{\bfseries \footnotesize #1}}

% For derivatives
\newcommand{\deriv}[1]{\frac{\mathrm{d}}{\mathrm{d}x} (#1)}

% For partial derivatives
\newcommand{\pderiv}[2]{\frac{\partial}{\partial #1} (#2)}

% Integral dx
\newcommand{\dx}{\mathrm{d}x}

% Alias for the Solution section header
\newcommand{\solution}{\textbf{\large Solución}}

% Probability commands: Expectation, Variance, Covariance, Bias
\newcommand{\E}{\mathrm{E}}
\newcommand{\Var}{\mathrm{Var}}
\newcommand{\Cov}{\mathrm{Cov}}
\newcommand{\Bias}{\mathrm{Bias}}

\begin{document}


\maketitle

\pagebreak

\begin{homeworkProblem}
¿Qué es la lógica?\\
\\
\solution\\
\\
La lógica es la disciplina la cual se encarga del estudio del razonamiento, por medio de reglas y técnicas determina si un argumento es válido o no.
\end{homeworkProblem}

\begin{homeworkProblem}
Menciona un ejemplo del uso de la lógica en Ciencias de la Computación.\\
\\
\solution\\
\\
El nivel menos abstracto dentro de una computadora está constituido por circuitos electrónicos que responden a diferentes señales eléctricas, siguiendo los patrones de la \textbf{lógica booleana}, estas compuertas lógicas devuelven un valor dependiendo de las entradas que le dan al sistema. Existen 8 compuertas lógicas, las cueles son: AND, OR, Inverter, Buffer, NAND, NOR, XOR y XNOR.
\end{homeworkProblem}

\begin{homeworkProblem}
¿Qué es una proposición?\\
\\
\solution\\
\\
Es un enunciado del cual se puede conocer su valor de verdad, es decir; podemos determinar si es verdadero o falso.
Con el contexto se puede determinar si un enunciado es una proposición, si no hay contexto entonces no es una proposición y las oraciones imperativas no son proposiciones.
\end{homeworkProblem}

\begin{homeworkProblem}
¿Qué es un predicado?\\
\\
\solution\\
\\
Un predicado es una expresión lingüística que puede conectarse con una o varias otras expresiones para formar una oración. Por ejemplo, en la oración "Marte es un planeta", la expresión "es un planeta" es un predicado que se conecta con la expresión "Marte" para formar una oración; entonces, un predicado expresara la(s) propiedad(es) que cumple el sujeto.
Los predicados en lógica matemática también son tratados como funciones
\end{homeworkProblem}

\pagebreak

\begin{homeworkProblem}
¿Qué es un argumento lógico?\\
\\
\solution\\
\\
Un argumento lógico es una secuencia de proposiciones seguidos de una conclusión. Y un argumento es considerado correcto o válido si es una tautología.
\end{homeworkProblem}

\begin{homeworkProblem}
¿Qué es una tautología?\\
\\
\solution\\
\\
Es una fórmula que siempre se evalúa como verdadera, para todas las posibles asignaciones de sus variables (la columna resultado de su tabla de verdad tiene sólo unos).
\end{homeworkProblem}

\begin{homeworkProblem}
¿Qué es una contradicción?\\
\\
\solution\\
\\
Es cuando una fórmula se evalúa como falsa en todos sus estados posibles (la columna resultado de su tabla de verdad tiene sólo ceros).
\end{homeworkProblem}

\begin{homeworkProblem}
¿Qué es una contingencia? (En Lógica)\\
\\
\solution\\
\\
Es cuando una fórmula no es tautología ni contradicción.
\end{homeworkProblem}

\begin{homeworkProblem}
¿Qué es la equivalencia? (En lógica)\\
\\
\solution\\
\\
Dos proposiciones o fórmulas lógicas son lógicamente equivalentes cuando toman el mismo valor de verdad en todas las posibles asignaciones de valores de sus variables, es decir; las tablas de verdad de las dos proposiciones dan el mismo resultado.
\end{homeworkProblem}

\pagebreak

\begin{homeworkProblem}
¿Qué es un modelo? (En lógica)\\
\\
\solution\\
\\
Un modelo de un lenguaje formal es una estructura donde las oraciones formales de la teoría (es decir, una cadena de signos matemáticos) son interpretables y por tanto las oraciones pueden considerarse como afirmaciones sobre el modelo. Si un modelo para un lenguaje formal satisface además una oración o una teoría (conjunto de oraciones), se llama modelo de una oración o teoría.\\
La idea es que un lenguaje formal es simplemente un inventario de los signos que vamos a usar, y un modelo de un lenguaje formal es una asignación de significado a cada uno de esos signos.
\end{homeworkProblem}

\begin{homeworkProblem}
¿Qué es una premisa?\\
\\
\solution\\
\\
Una premisa es cada una de las proposiciones anteriores a la conclusión de argumento.En un argumento válido, las premisas implican la conclusión, pero esto no es necesario para que una proposición sea una premisa: lo único relevante es su lugar en el argumento, no su rol. Al ser proposiciones, las premisas siempre afirman o niegan algo que puede ser verdadero o falso.
\end{homeworkProblem}

\begin{homeworkProblem}
¿Qué es una conclusión?\\
\\
\solution\\
\\
Una conclusión es una proposición al final de un argumento, después de las premisas. Con los criterios adecuados e pude afirmar que si las premisas son verdaderas entonces la conclusión también tiene que serlo necesariamente.
\end{homeworkProblem}

\pagebreak

\begin{homeworkProblem}
Da ejemplo de 5 expresiones (en lenguaje natural) que sean proposiciones y 3 que no lo sean.\\
\\
\solution\\
\\
Expresiones que son proposiciones:
\begin{enumerate}
    \item Cuatro es un número par.
    \item El rombo es un equilátero.
    \item Los leones no son voluminosos.
    \item Hoy es miércoles.
    \item La Tierra es plana.
\end{enumerate}
Expresiones que \textbf{no} son proposiciones:
\begin{enumerate}
    \item ¡Hola!
    \item Limpia la escalera.
    \item ¿Qué cayo del cielo?
\end{enumerate}
\end{homeworkProblem}

\begin{homeworkProblem}
Identifica en los siguientes enunciados cuáles son las proposiciones
atómicas y tradúcelas asignando utilizando variables proposicionales y los operadores que sean necesarios:

  \begin{enumerate}
      \item El libro es rojo.
      \item Juan y Mariana tienen sed.
      \item Como está muy nublado, va a llover. Por lo tanto no saldremos.
      \item El lápiz es rojo o es amarillo.
  \end{enumerate}

\end{homeworkProblem}


\begin{homeworkProblem} [41]
Cuántos renglones tiene la tabla de verdad de una proposición de $n$ variables? Justifica tu respuesta. (Si das una demostración formal se te da un punto extra).\\

\end{homeworkProblem}

\pagebreak

\cite{miquelteorias,lodepe}
\printbibliography

%\section{Bibliografía}
%\begin{itemize}
%\item https://www.ecured.cu/Logicamatematica % me manda 404
%\item https://es.wikipedia.org/wiki/PuertalConjuntodepuertasl % no encontre la pagina de esto
%\item https://sites.google.com/site/mathematicasdiscretesolutions/logica-de-po
%\item https://www.fing.edu.uy/~amiquel/fundamentos/teoriasymodelos.pdf
%\item https://es.wikipedia.org/wiki/Premisa\#:\~ :text=Enlogicaunapremisaelargumento
%\end{itemize}

\end{document}
