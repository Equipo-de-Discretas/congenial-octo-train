\documentclass{article}

\usepackage{fancyhdr}
\usepackage{extramarks}
\usepackage{amsmath}
\usepackage{amsthm}
\usepackage{amsfonts}
\usepackage{tikz}
\usepackage{csquotes}
\usepackage[plain]{algorithm}
\usepackage{algpseudocode}
\usepackage[spanish]{babel}
\usepackage[
backend=bibtex,
style=alphabetic,
sorting=ynt
]{biblatex}
\addbibresource{bibl.bib}
\usepackage{float}
\restylefloat{table}
\usepackage{hyperref}
\usepackage[utf8]{inputenc}

\usetikzlibrary{automata,positioning}

%
% Basic Document Settings
%

\topmargin=-0.45in
\evensidemargin=0in
\oddsidemargin=0in
\textwidth=6.5in
\textheight=9.0in
\headsep=0.25in

\linespread{1.1}

\pagestyle{fancy}
\lhead{\hmwkClass}
\chead{\hmwkTitle}
\rhead{\firstxmark}
\lfoot{\lastxmark}
\cfoot{\thepage}

\renewcommand\headrulewidth{0.4pt}
\renewcommand\footrulewidth{0.4pt}

\setlength\parindent{0pt}

%
% Create Problem Sections
%

\newcommand{\enterProblemHeader}[1]{
    \nobreak\extramarks{}{Problema \arabic{#1} continua en la p\'agina siguiente\ldots}\nobreak{}
    \nobreak\extramarks{Problema \arabic{#1} (continuaci\'on)}{Problema \arabic{#1} continua en la p\'agina siguiente\ldots}\nobreak{}
}

\newcommand{\exitProblemHeader}[1]{
    \nobreak\extramarks{Problema \arabic{#1} (continuaci\'on)}{Problema \arabic{#1} continua en la p\'agina siguiente\ldots}\nobreak{}
    \stepcounter{#1}
    \nobreak\extramarks{Problema \arabic{#1}}{}\nobreak{}
}

\setcounter{secnumdepth}{0}
\newcounter{partCounter}
\newcounter{homeworkProblemCounter}
\setcounter{homeworkProblemCounter}{1}
\nobreak\extramarks{Problema \arabic{homeworkProblemCounter}}{}\nobreak{}

%
% Homework Problem Environment
%
% This environment takes an optional argument. When given, it will adjust the
% problem counter. This is useful for when the problems given for your
% assignment aren't sequential. See the last 3 problems of this template for an
% example.
%
\newenvironment{homeworkProblem}[1][-1]{
    \ifnum#1>0
        \setcounter{homeworkProblemCounter}{#1}
    \fi
    \section{Problema \arabic{homeworkProblemCounter}}
    \setcounter{partCounter}{1}
    \enterProblemHeader{homeworkProblemCounter}
}{
    \exitProblemHeader{homeworkProblemCounter}
}

%
% Homework Details
%   - Title
%   - Due date
%   - Class
%   - Section/Time
%   - Instructor
%   - Author
%

\newcommand{\hmwkTitle}{Tarea\ \#2}
\newcommand{\hmwkDueDate}{Noviembre 11, 2020}
\newcommand{\hmwkClass}{Estructuras Discretas}
\newcommand{\hmwkClassTime}{}
\newcommand{\hmwkClassInstructor}{Profesora Alma Ar\'evalo Loyola}
\newcommand{\hmwkAuthorName}{\textbf{Sof\'ia Guadalupe Alatorre Méndez} \and \textbf{Diego Navarro Macías} \and \textbf{Edson Jair Morales Hernández}}


%
% Title Page
%

\title{
    \vspace{2in}
    \textmd{\textbf{\hmwkClass:\ \hmwkTitle}}\\
    \normalsize\vspace{0.1in}\small{Entregar\ el\ \hmwkDueDate\ a las 11:59pm}\\
    \vspace{0.1in}\large{\textit{\hmwkClassInstructor\ \hmwkClassTime}}
    \vspace{3in}
}


\author{\hmwkAuthorName}
\date{}

\renewcommand{\part}[1]{\textbf{\large Parte \Alph{partCounter}}\stepcounter{partCounter}\\}

%
% Various Helper Commands
%

% Useful for algorithms
\newcommand{\alg}[1]{\textsc{\bfseries \footnotesize #1}}

% For derivatives
\newcommand{\deriv}[1]{\frac{\mathrm{d}}{\mathrm{d}x} (#1)}

% For partial derivatives
\newcommand{\pderiv}[2]{\frac{\partial}{\partial #1} (#2)}

% Integral dx
\newcommand{\dx}{\mathrm{d}x}

% Alias for the Solution section header
\newcommand{\solution}{\textbf{\large Solución}}

% Probability commands: Expectation, Variance, Covariance, Bias
\newcommand{\E}{\mathrm{E}}
\newcommand{\Var}{\mathrm{Var}}
\newcommand{\Cov}{\mathrm{Cov}}
\newcommand{\Bias}{\mathrm{Bias}}

\begin{document}


\maketitle

\pagebreak

\begin{homeworkProblem}
¿Qué es la lógica?\\
\\
\solution\\
\\
La lógica es la disciplina la cual se encarga del estudio del razonamiento, por medio de reglas y técnicas determina si un argumento es válido o no.
\end{homeworkProblem}

\begin{homeworkProblem}
Menciona un ejemplo del uso de la lógica en Ciencias de la Computación.\\
\\
\solution\\
\\
El nivel menos abstracto dentro de una computadora está constituido por circuitos electrónicos que responden a diferentes señales eléctricas, siguiendo los patrones de la \textbf{lógica booleana}, estas compuertas lógicas devuelven un valor dependiendo de las entradas que le dan al sistema. Existen 8 compuertas lógicas, las cueles son: AND, OR, Inverter, Buffer, NAND, NOR, XOR y XNOR.
\end{homeworkProblem}

\begin{homeworkProblem}
¿Qué es una proposición?\\
\\
\solution\\
\\
Es un enunciado del cual se puede conocer su valor de verdad, es decir; podemos determinar si es verdadero o falso.
Con el contexto se puede determinar si un enunciado es una proposición, si no hay contexto entonces no es una proposición y las oraciones imperativas no son proposiciones.
\end{homeworkProblem}

\begin{homeworkProblem}
¿Qué es un predicado?\\
\\
\solution\\
\\
Un predicado es una expresión lingüística que puede conectarse con una o varias otras expresiones para formar una oración. Por ejemplo, en la oración <<Marte es un planeta>>, la expresión <<es un planeta>> es un predicado que se conecta con la expresión <<Marte>> para formar una oración; entonces, un predicado expresara la(s) propiedad(es) que cumple el sujeto.
Los predicados en lógica matemática también son tratados como funciones.
\end{homeworkProblem}

\pagebreak

\begin{homeworkProblem}
¿Qué es un argumento lógico?\\
\\
\solution\\
\\
Un argumento lógico es una secuencia de proposiciones seguidos de una conclusión. Y un argumento es considerado correcto o válido si es una tautología.
\end{homeworkProblem}

\begin{homeworkProblem}
¿Qué es una tautología?\\
\\
\solution\\
\\
Es una fórmula que siempre se evalúa como verdadera, para todas las posibles asignaciones de sus variables (la columna resultado de su tabla de verdad tiene sólo unos).
\end{homeworkProblem}

\begin{homeworkProblem}
¿Qué es una contradicción?\\
\\
\solution\\
\\
Es cuando una fórmula se evalúa como falsa en todos sus estados posibles (la columna resultado de su tabla de verdad tiene sólo ceros).
\end{homeworkProblem}

\begin{homeworkProblem}
¿Qué es una contingencia? (En Lógica)\\
\\
\solution\\
\\
Es cuando una fórmula no es tautología ni contradicción.
\end{homeworkProblem}

\begin{homeworkProblem}
¿Qué es la equivalencia? (En lógica)\\
\\
\solution\\
\\
Dos proposiciones o fórmulas lógicas son lógicamente equivalentes cuando toman el mismo valor de verdad en todas las posibles asignaciones de valores de sus variables, es decir; las tablas de verdad de las dos proposiciones dan el mismo resultado.
\end{homeworkProblem}

\pagebreak

\begin{homeworkProblem}
¿Qué es un modelo? (En lógica)\\
\\
\solution\\
\\
Un modelo de un lenguaje formal es una estructura donde las oraciones formales de la teoría (es decir, una cadena de signos matemáticos) son interpretables y por tanto las oraciones pueden considerarse como afirmaciones sobre el modelo. Si un modelo para un lenguaje formal satisface además una oración o una teoría (conjunto de oraciones), se llama modelo de una oración o teoría.\\
La idea es que un lenguaje formal es simplemente un inventario de los signos que vamos a usar, y un modelo de un lenguaje formal es una asignación de significado a cada uno de esos signos.
\end{homeworkProblem}

\begin{homeworkProblem}
¿Qué es una premisa?\\
\\
\solution\\
\\
Una premisa es cada una de las proposiciones anteriores a la conclusión de argumento.En un argumento válido, las premisas implican la conclusión, pero esto no es necesario para que una proposición sea una premisa: lo único relevante es su lugar en el argumento, no su rol. Al ser proposiciones, las premisas siempre afirman o niegan algo que puede ser verdadero o falso.
\end{homeworkProblem}

\begin{homeworkProblem}
¿Qué es una conclusión?\\
\\
\solution\\
\\
Una conclusión es una proposición al final de un argumento, después de las premisas. Con los criterios adecuados e pude afirmar que si las premisas son verdaderas entonces la conclusión también tiene que serlo necesariamente.
\end{homeworkProblem}

\pagebreak

\begin{homeworkProblem}
Da ejemplo de 5 expresiones (en lenguaje natural) que sean proposiciones y 3 que no lo sean.\\
\\
\solution\\
\\
Expresiones que son proposiciones:
\begin{enumerate}
    \item Cuatro es un número par.
    \item El rombo es un equilátero.
    \item Los leones no son voluminosos.
    \item Hoy es miércoles.
    \item La Tierra es plana.
\end{enumerate}
Expresiones que \textbf{no} son proposiciones:
\begin{enumerate}
    \item ¡Hola!
    \item Limpia la escalera.
    \item ¿Qué cayo del cielo?
\end{enumerate}
\end{homeworkProblem}

\begin{homeworkProblem}
Identifica en los siguientes enunciados cuáles son las proposiciones
atómicas y tradúcelas asignando utilizando variables proposicionales y los operadores que sean necesarios:\\
\\
\solution\\
  \begin{enumerate}
      \item El libro es rojo.\\
      La proposición atómica es <<El libro es rojo>>, no es posible separarla más. A esta proposición le asignaría la letra r y se quedaría como tal, ya que no hay operadores que agregar.\\

      \item Juan y Mariana tienen sed.\\
      Las proposiciones atómicas son <<Juan tiene sed>> representada con la variable p y <<Mariana tiene sed>> representada con q.\\
      Se traduciría como: \\
      $p \wedge \ q$

      \item Como está muy nublado, va a llover. Por lo tanto no saldremos.\\
      Las proposiciones atómicas son <<Como está muy nublado>> representada con a, <<Va a llover>> representada con b y <<Por lo tanto \textbf{no} saldremos>> representada como $\neg c$. \\
      Se \ traduciria \ como:\\
      $(a \rightarrow b) \rightarrow \neg c$

      \item El lápiz es rojo o es amarillo.\\
      Las proposiciones atómicas son <<El lápiz es rojo>> representada como k y <<El lápiz es amarillo>> como l. \\
      Se traduciría como: \\
      $k \vee l$
  \end{enumerate}

\end{homeworkProblem}

\begin{homeworkProblem}
 Menciona los operadores lógicos que cumplen con la propiedad conmutativa justificando tu respuesta. \\
\\
\solution\\
\\
Los operadores logicos que cumplen con esta propiedad son la conjunción: $\wedge$, la disyunción: $\vee$ y la doble implicación: $\longleftrightarrow$; por ejemplo supongamos que expresar $p \wedge q$, es lo mismo que expresar $q \wedge p$, lo mismo para la disyunción y la doble implicación, tenemos $p \vee q$ es igual a $q \vee p$, y por ultimo $p \longleftrightarrow q$ es igual a $q \longleftrightarrow p$
\end{homeworkProblem}

\begin{homeworkProblem}
Menciona los operadores lógicos que cumplen con la propiedad asociativa justificando tu respuesta. \\
\\
\solution\\
\\
Los operadores logicos que vendrían cumpliendo esta propiedad son los mismos que del problema anterior y sabemos que la asociatividad es de la forma: $(a + b)+c$ = $a + (b + c)$.
\\
En justificación de nuestra respuesta, si tenemos a $(p \wedge q) \wedge r$ es lo mismo que expresar $p \wedge (q \wedge r)$, y si tenemos $(p \vee q) \vee r$ es igual que $p \vee (q \vee r)$, y aplicamos lo mismo para la doble implicación $(p \Longleftrightarrow q) \Longleftrightarrow r$ $=$ $p \Longleftrightarrow (q \Longleftrightarrow r)$


\end{homeworkProblem}


\begin{homeworkProblem}
  Coloca los paréntesis en las siguientes expresiones de tal manera que
  respeten la precedencia de los operadores lógicos \\
 \begin{itemize}
    \item $s \wedge r \wedge q \vee t$\\
    \solution\\
    \\
    $(s \wedge r) \wedge (q \vee t)$


    \item $a \vee b \vee c \vee d \vee e$
    \\
    \solution \\
    \\
    $((a \vee b) \vee (c \vee d)) \vee e$

    \item $p \longrightarrow q \wedge s \longrightarrow r \vee a$
    \\
    \solution\\
    \\
    $(p \longrightarrow (q \wedge s)) \longrightarrow (r \vee a)$

    \item $\neg s \longleftrightarrow t \longrightarrow q \longrightarrow p$ \\
     \\
     \solution\\
     \\
     $(\neg s \longleftrightarrow t) \longrightarrow (q \longrightarrow p)$

    \item $P(x) \longrightarrow R(n,k) \wedge 1=1 \longrightarrow P(1) \vee P(0) \longleftrightarrow R(0,1)$\\
    \\
    \solution\\
    \\
    $P(x) \longrightarrow ((R(n,k) \wedge 1=1) \longrightarrow (P(1) \vee P(0)) \longleftrightarrow R(0,1))$

    \item $P(n) \longrightarrow P(n +1) \longrightarrow P(n+2) \longrightarrow P(n+3) \longrightarrow P(n+4) \longrightarrow
    P(n+5)$\\
    \\
    \solution
    \\
    $(((((P(n) \longrightarrow P(n +1)) \longrightarrow P(n+2)) \longrightarrow P(n+3)) \longrightarrow P(n+4)) \longrightarrow
    P(n+5))$

    \item $\neg p \vee P(n) \wedge P(n+1) \longrightarrow \neg(p \vee s)\wedge w \longrightarrow R(n,n+1)$\\
    \\
    \solution\\
    \\
    $(((\neg p \vee P(n)) \wedge P(n+1)) \longrightarrow \neg((p \vee s)\wedge w)) \longrightarrow R(n,n+1)$

    \end{itemize}

\end{homeworkProblem}

\begin{homeworkProblem}[29]
  Justifica por qué no son equivalentes las siguientes expresiones

    \begin{table}[H]
        \centering
        \begin{tabular}{c c c}
        $\forall x \exists y (y+x = 0)$ & y &$ \exists y \forall x(y+x = 0)$  \\
        \end{tabular}
    \end{table}

    \solution\\
    \\
    Cambiar los cuantificadores, no es normalmente aceptado, los dos enunciados
    son validos, pero no quiere eso decir que son equivalentes, o congruentes,
    dado esto, podemos ver que al cambiar $\forall$ y $\exists$ no llega a un
    resultado congruente:\\
      \[
        \forall x \exists y (y+x = 0)  \not\equiv  \exists y \forall x(y+x = 0)
      \]
  Por ejemplo:\\
  \\
  Sea $Q(x,y)$  como "x quiere a y". Siendo of $x,y: personas$

    \begin{itemize}
    \item El lado izquierdo seria $\forall x\exists yQ(x,y)$ : "Todos quieren a alguien". Toda persona quiere a alguna persona.

    \item El lado derecho, siendo: $\exists y\forall xQ(x,y)$ : "Existe alguna persona que queire a todos". Alguien quiere a todas las personas.
    \end{itemize}

    Por esto diria que no son lo mismo.
\end{homeworkProblem}

\begin{homeworkProblem}
En las siguientes proposiciones identifica las variables libres, las 
    variables ligadas y los alcances de los cuantificadores. No omitas
    el introducir paréntesis donde sea necesario. \\
    \\
\begin{itemize}
        \item $\forall x\ x \leq y \longrightarrow \exists z\ x + z = y$\\
        \\
        \solution\\
        En este caso las variables ligadas son $x$ y $z$ ya que estas se encuentran en los cuantificadores; y la variable libre es y ya que no esta siendo cuantificada.\\
        \\
        \\
        El alcance del cuantificador $\forall x$ es $((x \leq y) \longrightarrow \exists z\ (x + z = y))$ y el del $\exists z$
        es $(x + z = y)$
        \\
        
        \item $\forall \epsilon > 0\  \exists \delta > 0 (|x-y|<\delta \longrightarrow
        |f(x) -f(y)| < \epsilon$)\\
        \\
        \solution\\
        En este problema nuestras variables ligadas son $\epsilon$ y $\delta$ dejando como variables libres a $x$ y $y$\\
        \\
        
        
        El alcance del $\forall \epsilon > 0$ es $(\exists \delta > 0 (|x-y|<\delta \longrightarrow
        |f(x) -f(y)| < \epsilon)$) y el de $\exists \delta > 0$ es $(|x-y|<\delta \longrightarrow
        |f(x) -f(y)| < \epsilon)$
        
        \item $\forall x \in \mathbf{N}\ \exists y \in \mathbf{N}\ suc(x) = y$ \\
         \\
         \solution\\
         En este caso tenemos a todos los variables siendo cuantificadas que aqui son $x$ y $y$.\\
         \\
         El alcance del cuantificador $\forall x \in \mathbf{N}$ es $(\exists y \in \mathbf{N}\ (suc(x) = y))$  y el de $\exists y \in \mathbf{N}$ es $(suc(x) = y)$
        
         
        \item $\exists x \forall y \exists a(z + 2w = 4y \vee 2s = y^3+7)$ \\
        \\
        \solution\\
        Nuestras variables ligadas son $x$, $y$ y $a$ y nuestras variables libres son $z$, $w$ y $s$. \\
        
        El alcance del cuantificador $\exists x$ es $\forall y \exists a(z + 2w = 4y \vee 2s = y^3+7)$, el de $\forall y$ es $\exists a(z + 2w = 4y \vee 2s = y^3+7)$ y el de $\exists a$ es $(z + 2w = 4y \vee 2s = y^3+7)$.
        
        \item $\forall \epsilon(\epsilon > 0 \longrightarrow \exists \delta > 0 \vee \epsilon + y = x)$ \\
        \\
        \solution\\
        Las variables ligadas son $\epsilon$ y $\delta$ mientras que las libres son $x$ y $y$\\
        \\
        El alcance de $\forall \epsilon$ es $(\epsilon > 0 \longrightarrow ((\exists \delta > 0) \vee (\epsilon + y = x)))$ y el de $(\exists \delta > 0)$ sería unicamente el ya que va seguido de una disyunción.
        
        \item $\forall x \forall y \exists z R(x,n) \longrightarrow P(z)$\\
        \\
        \solution\\
        \\
        En este problema las variables ligadas son $x$, $y$ y $z$ y la variable libre sería unicamente la $n$ \\
        \\
        En este el alcance de $\forall x$ es $\forall y \exists z (R(x,n) \longrightarrow P(z))$, el de $\forall y$ es $\exists z (R(x,n) \longrightarrow P(z))$ y el de $\exists z$ es $(R(x,n) \longrightarrow P(z))$.  \\
        
        \item $P(0) \wedge \forall k \ (P(k) \longrightarrow P(suc(k))) \longrightarrow \forall n  P(n)$\\
        
        \solution
        \\
        En este caso las variables ligadas son $k$ y $n$ y no hay ninguna variable libre.\\
        \\
        
        En el caso de $\forall k$ su alcance seria  $\ (P(k) \longrightarrow P(suc(k))) \longrightarrow \forall n  P(n)$, mientras que el de $\forall n$ es solo $P(n)$ 
        
    \end{itemize}
\end{homeworkProblem}

\begin{homeworkProblem}[41]
Cuántos renglones tiene la tabla de verdad de una proposición de $n$ variables? Justifica tu respuesta. (Si das una demostración formal se te da un punto extra).\\
\\
\solution\\
\\
Toda tabla de verdad tiene $2^n$ renglones, donde la $n$ corresponde al número de variables proposicionales que aparecen en la fórmula.\\
Esta fórmula sirve para conocer el número de combinaciones que existen de asignaciones de valores de verdad a cada una de las variables, siempre y cuando se cumpla con que los valores de verdad sean sólo dos -verdadero o falso- y que existan asignaciones de valores de verdad a cada una de las variables.
\end{homeworkProblem}

\begin{homeworkProblem}
    Pasa las siguientes expresiones a forma normal disyuntiva.
        \begin{enumerate}
            \item $(p \longrightarrow q) \vee (q \longrightarrow p)$
            \item $(p \longrightarrow r) \wedge (q \longrightarrow r)$
            \item $(\neg r) \longrightarrow (p \longrightarrow q)$
            \item $(\neg p) \longleftrightarrow q$
        \end{enumerate}
\end{homeworkProblem}

\begin{homeworkProblem}
  Existe un algoritmo\footnote{De tiempo polinomial} que dada una expresión
    lógica $\varphi$ de como resultado una expresión $\varphi'$ equivalente pero con
    la mínima cantidad de variables y operadores posibles?\footnote{Esto quiere decir que
    $\varphi'$ es mínima.}
    \\
    \solution\\
    \\
    No existe, podemos ver en esta referencia.\cite{brayton1984logic}
\end{homeworkProblem}

\begin{homeworkProblem}
  \textbf{Cieto o falso:} Sea $\Gamma$ un conjunto de formulas.
  Si para cualquier subconjunto de $\Gamma'\subseteq \Gamma$ se tiene que $\Gamma'$
  es satisfacible, entonces $\Gamma$ es satisfacible.
  Si no así, muestra un ejemplo. Si la respuesta es si, justifica.
  \\
  \solution\\
  \\
  Cierto.
\end{homeworkProblem}
\pagebreak

\cite{miquelteorias,lodepe}
\printbibliography


\end{document}
